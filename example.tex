% Beim Laden der documentclass können zwar Schrift- und Papiergröße angegeben werden, allerdings überschreibt das Paket fth-lsa
\documentclass{scrreport}

\usepackage{fth/fth-lang}
% Paket für Hebräisch und Griechisch, muss unbedingt vor den anderen Paketen geladen werden.
% Erfordert Installation der Fonts SBL Hebrew und Greek: 
% https://www.sbl-site.org/educational/BiblicalFonts_SBLHebrew.aspx
% https://www.sbl-site.org/educational/BiblicalFonts_SBLGreek.aspx
% Außerdem sind auch Targum-Aramäisch (mit babylonischer Punktierung) mit dem Font EzraBab
% und Syrisch mit dem Font Estrangelo Edessa nutzbar.
% Griechisch, Hebräisch, Targum-Aramäisch und Syrisch sind jeweils über die Paket-Optionen
% grk, heb, tgaram und syr ein- und ausschaltbar, also z.B.
% \usepackage[grk=false,syr=true]{fth-lang}.
% Dabei sind Griechisch und Hebräisch per Default aktiviert, Targum-Aramäisch und Syrisch nicht.

\usepackage{fth/fth-lsa}
% Verfügbare Optionen für das Paket fth-lsa
% \usepackage[bound,thesis,tre]{fth/fth-lsa}
%      bound:  für gebundene Arbeiten (breiterer Rand)
%      thesis: für Abschlussarbeiten (andere Titelseite und Fußnoten kapitelweise nummeriert), 
%              siehe example_thesis.tex
%      tre:    Formattiert Bibelstellen anhand TRE anstelle von Loccumer Richtlinien (siehe dazu Anmerkung unten)
%      title-style=sc: verwendet für Titel und Untertitel auf dem Deckblatt Kapitälchen antatt Großbuchstaben
%      lecture-notes: für Mitschriften: kleinere Seitenränder und Zeilenabstände


% Anmerkung zu Bibelstellen: Die FTH-Leitlinien fordern zwar eine Fomrattierung entsprechend den
% Loccumer Richtlinien, die in Abschnitt 6.1 der Leitlinien angegebenen Abkürzungen entsprechen
% allerdings gar nicht den Loccumer Richtlinien (das Buch 1.Mose müsste z.B. Gen abgekürzt werden,
% nicht 1Mose). Bei Benutzung von \bibleverse formattiert das Paket fth-lsa Bibelstellen korrekt
% entsprechend den echten Loccumer Richtlinien, nicht jedoch wie in 6.1 der FTH-Leitlinien. Wem
% das zu riskant ist, kann die Option tre verwenden, die stattdessen entsprechend der TRE formattiert,
% was die FTH-Leitlinien ebenfalls erlauben.


\usepackage{fth/fth-nt-utils}
% Paket mit hilfreichen Befehlen für die Arbeit an neutestamentlichen Texten, s.u. für Beispiele

\usepackage[authorsc]{fth/fth-bib}
% Verfügbare Optionen für fth-bib 
%     authorsc: Wenn gesetzt, wird Autor bei \citeauthor und in der Bibliographie in Small Caps
%               (Kapitälchen) gedruckt

\addbibresource{literature.bib}

% Um am Ende ein Bibelstellenverzeichnis zu generieren
\usepackage{imakeidx}
\makeindex[title=Bibelstellenverzeichnis]

\usepackage{lipsum}

\begin{document}

% \title, \author und \date müssen immer gesetzt sein, alle anderen Felder sind optional
\title{Umsetzung der \enquote{Leitlinien für schriftliche Arbeiten an der FTH}}
\subtitle{Ein Beispieldokument mit \grk{κοινὴ} und \heb{עִבְרִית}}
\thesistype{Übung}
\course{Einführung ins wissenschaftliche Arbeiten}
% Alternativ bei mehreren Fächern:
%\courses{\enquote{Einführung in die Praktische Theologie} und \enquote{Grundlagen des missionarischen Gemeindeaufbaus}}
\module{Propädeutikum Theologie}
\semester{1. Semester, B.A.}
\lecturer{Prof. Dr. Z}
\author{Berta Beispiel}
\date{25. November 2022}

\maketitle

\tableofcontents

\chapter{Ein Kapitel}
\label{chap:firstchapter}
\section{Ein Abschnitt}
\lipsum

\section{Noch ein Abschnitt}
In diesem Abschnitt wird etwas in Anführungszeichen gesetzt: \blockquote{In diesem Anführungszeichen gibt es noch welche \enquote{nämlich hier}}. Und jetzt kommt ein Zitat über ein paar mehr Zeilen: \blockquote{\lipsum[1]} Und weiter geht's mit dem Absatz, ohne Einrückung. Das klappt ja hervorragend. \lipsum[2]. \blockquote{Hier ein Zitat, das über fast 3 Zeilen geht und daher nicht als Block dargestellt werden sollte. Es ist wie es ist, und es kommt wie es kommt. So kommt es nämlich immer. Ein bisschen Text brauch ich noch, und noch ein bisschen, und wenn jetzt noch was}

\sloppy
Bibelstellen sollten entsprechend den Loccumer Richtlinien oder der TRE abgekürzt werden. Hier kommen ein paar, die nach den Loccumer Richtlinien formattiert sind: \ibibleverse{Gen}(1:3), \ibibleverse{2Koen}{17}, \ibibleverse{Offb}(21:). Wenn man das Buch nicht abkürzen will, muss man \texttt{biblerefformat} umstellen: {\biblerefformat{lang} \ibibleverse{Mt}(1:)}

\fussy
\section{Fußnoten}
In diesem Abschnitt kommt mal eine Fußnote.\footnote{Das hier ist eine Fußnote. Sie geht über mehrere Zeilen, und sie sollten einen einzeiligen Zeilenabstand haben. \lipsum[1]} Jetzt nochmal einiger Text: \lipsum[1-9]\footnote{Das hier ist eine weitere Fußnote im selben Kapitel. Sie sollte fortlaufend numeriert werden.}

\chapter{Noch ein Kapitel mit weiteren Fußnoten und tollen Sprachen}
Das hier ist ein neues Kapitel.\footnote{Das hier ist eine Fußnote im neuen Kapitel. Sie sollte trotzdem weiter numeriert werden und nicht die Nummer 1 erhalten}. Außerdem noch ein Rückverweis auf \autoref{chap:firstchapter}

\section{Überschrift mit Griechischen Buchstaben: \grk{λόγος}}
\label{sec:greek}
Das funktioniert ja hervorragend! Und noch ein bisschen Griechisch im Text: \grk{Ἐν ἀρχῇ ἦν ὁ λόγος, καὶ ὁ λόγος ἦν \textbf{πρὸς τὸν θεόν}, καὶ \textit{θεὸς} ἦν ὁ λόγος.}

\section{Überschrift mit Hebräischen Buchstaben: \heb{בְּרֵאשִׁ֖ית}}
Etwas hebräischer Text: \heb{בְּרֵאשִׁ֖ית בָּרָ֣א אֱלֹהִ֑ים \textbf{אֵ֥ת} הַשָּׁמַ֖יִם וְאֵ֥ת הָאָֽרֶץ׃}. In \autoref{sec:greek} gab's Griechisch.

% Der folgende Abschnitt enthält babylonische Punktierung, die nur funktioniert, wenn die Schriftart EzraBab installiert ist oder EzraBab.ttf im Verzeichnis liegt. Wenn das der Fall ist, dann einfach die folgenden Zeilen einkommentieren.
%\section{Überschrift mit babylonischer Punktierung (Targum-Aramäisch): \tgaram{א֘ד֘ם}}
%Und hier Gen 2,15 nach Targum Onqelos: \tgaram{ו֝דב֨ר יוי א֓ל֯ה֜ים י֘ת א֘ד֘ם ו֓א֨שר֜י֞יה ב֓ג֜ינ֓ת֘א ד֓ע֞ד֨ן ל֓מִפל֓ח֨ה ו֝למ֜ט֓ר֨ה׃}

% Der folgende Abschnitt enthält Syrische Schrift, die nur funktioniert wenn die Schfirtart Estrangelo Edessa installiert ist oder estre.ttf im Verzeichnis liegt. Wenn das der Fall ist, einfach die folgenden Zeilen einkommentieren.
% \section{Überschrift mit Syrisch: \syr{ܫܪ̈ܒܬܗܘܢ ܫܢܬܐ ܢܗ̈ܘܝܢ܂ ܘܒܨܦܪܐ ܐܝܟ ܥܘܦܝܐ ܬܚܠܦ} }
% Und Text in Syrisch: \syr{ܫܪ̈ܒܬܗܘܢ ܫܢܬܐ ܢܗ̈ܘܝܢ܂ ܘܒܨܦܪܐ ܐܝܟ ܥܘܦܝܐ ܬܚܠܦ} 

\chapter{Kapitel mit Zitaten}
Hier könnte jetzt ein sinnvoller Text stehen. Aber es geht ja nur um die korrekte Form. Daher steht hier einfach irgendetwas. Jetzt kommt ein direktes Zitat. \citeauthor*{friesen} schreibt zum Thema Berufung: \blockcquote[][330]{friesen}{Rather than waiting for some kind of mystical \enquote{call} from God, every believer should respond to the revealed will of God by giving serious consideration to becoming a cross-cultural missionary.} Jetzt kommt noch ein Zitat aus demselben Werk: \blockcquote[][330]{friesen}{We don't need a call -- we've already been commissioned.} Weil dieses Zitat aus demselben Werk kommt und auf derselben Seite steht, sollte die Fußnote es einfach mit \enquote{ebd.} erwähnen.

Jetzt wäre mal ein Zitat aus einem Nachschlagewerk dran: \blockcquote[][80]{rienecker}{Gott selbst ist es, dessen Ruf die Menschen trifft und damit in eine Entscheidung stellt.} Der Glaube des Menschen sei dann die Antwort auf Gottes Heilswerk \autocite[Vgl.][]{rienecker}. Der letzte Satz war eine Paraphrase und kein direktes Zitat.

Und nun ein etwas längeres Zitat, das in Blocksatz gestellt wird, aus einer Fachzeitschrift: \blockcquote[][]{dahl}{Es ist also nicht länger möglich, Kulturen geographisch oder thematisch einzugrenzen. Menschen sind vielleicht mehr denn je zu etwas geworden, was man kulturelle Hybride nennen kann. Das sind sie wahrscheinlich immer gewesen, aber es fällt uns jetzt besonders auf, wenn uns die Globalisierung und Internationalisierung alles durch\-einander bringt.}

Völlig aus dem Zusammenhang gerissen, aber weil es ein Sammelband ist, den ich gerade da habe und zitieren kann, schreibt \citeauthor*{hilbrands}: \blockcquote[][49]{hilbrands}{Wohl in keiner anderen theologischen Disziplin sind die Einflüsse der Bibelkritik so nachhaltig gewesen wie im Bereich des Alten Testaments.}

Auch die Weiten des Internets bieten vieles, was man zitieren kann: \blockcquote[][]{strecker}{Unter \enquote{Berufung} versteht man gemeinhin die \textelp{} Indienstnahme für bestimmte Aufgaben.}

\chapter{Kapitel mit NT-Späßen}
Das Paket \texttt{fth-nt-utils} bietet einige hilfreiche Befehle für die Arbeit mit dem Neuen Testament, insbesondere für das NT-Seminar im 3. Semester. So enthält es Befehle für die Symbole für den Nestle-Aland (\na), die LXX (\lxx), den Mehrheitstext (\majoritytext) und Papyri (\papyrus{46}). Außerdem können textkritische Symbole auf zwei Arten angezeigt werden, entweder direkt mit dem Unicode-Symbol \grk{⸀ ⸂ ⸃ ⸆ ° ⸉ ⸊} oder mit besonderen Befehlen \tcOmissionStart{} \tcOmissionEnd. Für alle verfügbaren Befehle bitte direkt den Code des Pakets konsultireren.

Für Textkritik gibt es auch Befehle, um Tabellen textkritischer Varianten im Format für das NT-Seminar zu erstellen:
\begin{table}[ht]
    \centering
    \begin{textvarianttable}
        \textvariant{
            \grk{ὅς} (\textit{textus})
        }{
            --
        }{
            \emph{\witnesscat{א}{4}{I}};
            \emph{\witnesscat{A}{5}{I}};
            \witnesscat{C}{5}{II};
            \witnesscat{K}{9}{V};
            \witnesscat{L}{9}{V};
            \witnesscat{P}{9}{III};
            \witnesscat{Ψ}{9/10}{III};
            \witnesscat{075}{10}{III}; 
            \witnesscat{0208}{6}{III};
            \witnesscat{0278}{9}{?}
        }{
            \witnesscat{33}{9}{II};
            \witnesscat{81}{11}{II};
            \witnesscat{104}{11}{III};
            \witnesscat{365}{13}{III};
            \witnesscat{630}{14}{III};
            \emph{\witnesscat{1175}{11}{I}};
            \witnesscat{\witness{1241}{s}}{12}{III};
            \witnesscat{1505}{11}{III};
            \emph{\witnesscat{1739}{10}{I}};
            \witnesscat{1881}{14}{II};
            \witnesscat{2464}{9}{II};
            \majoritytext
        }{
            lat
        }{
            \witnesscat{Ambst}{4}{}
        }
    
    
        \textvariant{
            \grk{ὅ}
        }{
            \emph{\witnesscat{\papyrus{46}}{2/3}{I}}
        }{
            \witnesscat{B}{4}{III};
            \witnesscat{D}{6}{II};
            \witnesscat{F}{9}{II};
            \witnesscat{G}{9}{III}
        }{
            --
        }{
            --
        }{
            \witnesscat{\witness{Hil}{pt}}{4}{}
        }
    \end{textvarianttable}

    \caption{Varianten für Kol 2,10 \grk{⸀ὅς}}
\end{table}

Für synoptische Vergleiche gibt es ebenfalls Befehle:
\begin{table}[ht]
    \begin{synopticcomparisontable}{Mt 18:2-5}{Mk 9:36-37}{Lk 9:47b-48}
         \synopticentry{
             \grk{2 \synspecial{καὶ προσκαλεσάμενος} \synall{παιδίον ἔστησεν αὐτὸ} \synmtmk{ἐν μέσῳ αὐτῶν} 3 καὶ \synall{εἶπεν·}}
         }{
             \grk{36 καὶ \synmklk{λαβὼν} \synall{παιδίον ἔστησεν αὐτὸ} \synmtmk{ἐν μέσῳ αὐτῶν} \synspecial{καὶ ἐναγκαλισάμενος αὐτὸ} \synall{εἶπεν} αὐτοῖς·}
         }{
             \grk{\synmklk{ἐπιλαβόμενος} \synall{παιδίον ἔστησεν αὐτὸ} \synspecial{παρ’ ἑαυτῷ} 48 καὶ \synall{εἶπεν} αὐτοῖς·}
         }
 
         \synopticentry{
             \grk{\synspecial{ἀμὴν λέγω ὑμῖν, ἐὰν μὴ στραφῆτε καὶ γένησθε ὡς τὰ παιδία, οὐ μὴ εἰσέλθητε εἰς τὴν βασιλείαν τῶν οὐρανῶν. 4 ὅστις οὖν ταπεινώσει ἑαυτὸν ὡς τὸ παιδίον τοῦτο,} \synmtlk{οὗτός ἐστιν ὁ μείζων} \synspecial{ἐν τῇ βασιλείᾳ τῶν οὐρανῶν.}}
         }{
         }{
         }
 
         \synopticentry{
             \grk{5 καὶ \synall{ὃς ἐὰν δέξηται} \synmtmk{ἓν} \synall{παιδίον} \synmtmk{τοιοῦτο} \synall{ἐπὶ τῷ ὀνόματί μου, ἐμὲ δέχεται.}}
         }{
             \grk{37 \synall{ὃς ἂν} \synmtmk{ἓν τῶν τοιούτων} \synall{παιδίων δέξηται ἐπὶ τῷ ὀνόματί μου, ἐμὲ δέχεται·} \synmklk{καὶ ὃς ἂν ἐμὲ δέχηται,} \synspecial{οὐκ ἐμὲ} \synmklk{δέχεται} \synspecial{ἀλλὰ} \synmklk{τὸν ἀποστείλαντά με}}
         }{
             \grk{\synall{ὃς ἐὰν δέξηται} \synspecial{τοῦτο} \synall{τὸ παιδίον ἐπὶ τῷ ὀνόματί μου, ἐμὲ δέχεται·} \synmklk{καὶ ὃς ἂν ἐμὲ δέξηται, δέχεται τὸν ἀποστείλαντά με·}}
         }
 
         \synopticentry{
         }{
         }{
             \grk{\synspecial{ὁ γὰρ μικρότερος ἐν πᾶσιν ὑμῖν ὑπάρχων} \synmtlk{οὗτός ἐστιν μέγας}}
         }
    \end{synopticcomparisontable}
    \caption{Synoptischer Vergleich zum Rangstreit}
    \label{tab:synopticcomparison}
 \end{table}

\newpage
\printbibliography[title=Literaturverzeichnis]

\newpage
\printindex

\newpage
\statementofauthorship[8. März 2024]
% Mit dem optionalen Parameter kann man das Datum für die Selbstständigkeitserklärung angeben.
% Per Default wird das Datum verwendet, das mit \date{} auf der Titelseite festgelegt wurde


\end{document}
