% Beim Laden der documentclass können zwar Schrift- und Papiergröße angegeben werden, allerdings überschreibt das Paket fth-lsa
\documentclass{scrreport}

\usepackage[ngerman]{babel}

\usepackage{fth/fth-lsa}
% Verfügbare Optionen für das Paket fth-lsa
% \usepackage[bound,thesis,tre]{fth/fth-lsa}
%      bound:  für gebundene Arbeiten (breiterer Rand)
%      thesis: für Abschlussarbeiten (andere Titelseite und Fußnoten kapitelweise nummeriert), 
%              siehe example_thesis.tex
%      tre:    Formattiert Bibelstellen anhand TRE anstelle von Loccumer Richtlinien (siehe dazu Anmerkung unten)
%      title-style=sc: verwendet für Titel und Untertitel auf dem Deckblatt Kapitälchen antatt Großbuchstaben


% Anmerkung zu Bibelstellen: Die FTH-Leitlinien fordern zwar eine Fomrattierung entsprechend den
% Loccumer Richtlinien, die in Abschnitt 6.1 der Leitlinien angegebenen Abkürzungen entsprechen
% allerdings gar nicht den Loccumer Richtlinien (das Buch 1.Mose müsste z.B. Gen abgekürzt werden,
% nicht 1Mose). Bei Benutzung von \bibleverse formattiert das Paket fth-lsa Bibelstellen korrekt
% entsprechend den echten Loccumer Richtlinien, nicht jedoch wie in 6.1 der FTH-Leitlinien. Wem
% das zu riskant ist, kann die Option tre verwenden, die stattdessen entsprechend der TRE formattiert,
% was die FTH-Leitlinien ebenfalls erlauben.

\usepackage[authorsc]{fth/fth-bib}
% Verfügbare Optionen für fth-bib 
%     authorsc: Wenn gesetzt, wird Autor bei \citeauthor und in der Bibliographie in Small Caps
%               (Kapitälchen) gedruckt

\addbibresource{literature.bib}

% Um am Ende ein Bibelstellenverzeichnis zu generieren
\usepackage{imakeidx}
\makeindex[title=Bibelstellenverzeichnis]

\usepackage{lipsum}

\begin{document}

% \title, \author und \date müssen immer gesetzt sein, alle anderen Felder sind optional
\title{Umsetzung der \enquote{Leitlinien für schriftliche Arbeiten an der FTH}}
\subtitle{Ein Beispieldokument}
\thesistype{Übung}
\course{Einführung ins wissenschaftliche Arbeiten}
% Alternativ bei mehreren Fächern:
%\courses{\enquote{Einführung in die Praktische Theologie} und \enquote{Grundlagen des missionarischen Gemeindeaufbaus}}
\module{Propädeutikum Theologie}
\semester{1. Semester, B.A.}
\lecturer{Prof. Dr. Z}
\author{Berta Beispiel}
\date{25. November 2022}

\maketitle

\tableofcontents

\chapter{Ein Kapitel}
\section{Ein Abschnitt}
\lipsum

\section{Noch ein Abschnitt}
In diesem Abschnitt wird etwas in Anführungszeichen gesetzt: \blockquote{In diesem Anführungszeichen gibt es noch welche \enquote{nämlich hier}}. Und jetzt kommt ein Zitat über ein paar mehr Zeilen: \blockquote{\lipsum[1]} Und weiter geht's mit dem Absatz, ohne Einrückung. Das klappt ja hervorragend. \lipsum[2]. \blockquote{Hier ein Zitat, das über fast 3 Zeilen geht und daher nicht als Block dargestellt werden sollte. Es ist wie es ist, und es kommt wie es kommt. So kommt es nämlich immer. Ein bisschen Text brauch ich noch, und noch ein bisschen, und wenn jetzt noch was}

\sloppy
Bibelstellen sollten entsprechend den Loccumer Richtlinien oder der TRE abgekürzt werden. Hier kommen ein paar, die nach den Loccumer Richtlinien formattiert sind: \ibibleverse{Gen}(1:3), \ibibleverse{2Koen}{17}, \ibibleverse{Offb}(21:). Wenn man das Buch nicht abkürzen will, muss man \texttt{biblerefformat} umstellen: {\biblerefformat{lang} \ibibleverse{Mt}(1:)}

\fussy
\section{Fußnoten}
In diesem Abschnitt kommt mal eine Fußnote.\footnote{Das hier ist eine Fußnote. Sie geht über mehrere Zeilen, und sie sollten einen einzeiligen Zeilenabstand haben. \lipsum[1]} Jetzt nochmal einiger Text: \lipsum[1-9]\footnote{Das hier ist eine weitere Fußnote im selben Kapitel. Sie sollte fortlaufend numeriert werden.}

\chapter{Noch ein Kapitel mit weiteren Fußnoten}
Das hier ist ein neues Kapitel.\footnote{Das hier ist eine Fußnote im neuen Kapitel. Sie sollte trotzdem weiter numeriert werden und nicht die Nummer 1 erhalten}

\section{Überschrift mit Griechischen Buchstaben: λόγος}
Das funktioniert ja hervorragend! Und noch ein bisschen Griechisch im Text: πνευματος.

\chapter{Kapitel mit Zitaten}
Hier könnte jetzt ein sinnvoller Text stehen. Aber es geht ja nur um die korrekte Form. Daher steht hier einfach irgendetwas. Jetzt kommt ein direktes Zitat. \citeauthor*{friesen} schreibt zum Thema Berufung: \blockcquote[][330]{friesen}{Rather than waiting for some kind of mystical \enquote{call} from God, every believer should respond to the revealed will of God by giving serious consideration to becoming a cross-cultural missionary.} Jetzt kommt noch ein Zitat aus demselben Werk: \blockcquote[][330]{friesen}{We don't need a call -- we've already been commissioned.} Weil dieses Zitat aus demselben Werk kommt und auf derselben Seite steht, sollte die Fußnote es einfach mit \enquote{ebd.} erwähnen.

Jetzt wäre mal ein Zitat aus einem Nachschlagewerk dran: \blockcquote[][80]{rienecker}{Gott selbst ist es, dessen Ruf die Menschen trifft und damit in eine Entscheidung stellt.} Der Glaube des Menschen sei dann die Antwort auf Gottes Heilswerk \autocite[Vgl.][]{rienecker}. Der letzte Satz war eine Paraphrase und kein direktes Zitat.

Und nun ein etwas längeres Zitat, das in Blocksatz gestellt wird, aus einer Fachzeitschrift: \blockcquote[][]{dahl}{Es ist also nicht länger möglich, Kulturen geographisch oder thematisch einzugrenzen. Menschen sind vielleicht mehr denn je zu etwas geworden, was man kulturelle Hybride nennen kann. Das sind sie wahrscheinlich immer gewesen, aber es fällt uns jetzt besonders auf, wenn uns die Globalisierung und Internationalisierung alles durch\-einander bringt.}

Völlig aus dem Zusammenhang gerissen, aber weil es ein Sammelband ist, den ich gerade da habe und zitieren kann, schreibt \citeauthor*{hilbrands}: \blockcquote[][49]{hilbrands}{Wohl in keiner anderen theologischen Disziplin sind die Einflüsse der Bibelkritik so nachhaltig gewesen wie im Bereich des Alten Testaments.}

Auch die Weiten des Internets bieten vieles, was man zitieren kann: \blockcquote[][]{strecker}{Unter \enquote{Berufung} versteht man gemeinhin die \textelp{} Indienstnahme für bestimmte Aufgaben.}

\newpage
\printbibliography[title=Literaturverzeichnis]

\newpage
\printindex

\newpage
\statementofauthorship


\end{document}
