% scrartcl ist eine geeignete Klasse für bspw. Handouts.
\documentclass{scrartcl}

% Wir wollen griechisch und hebräisch nutzen, daher "fth-lang".
% Voraussetzung dafür ist, dass die Fonts "SBL Hebrew" "SBL Greek" installiert sind!
% https://www.sbl-site.org/educational/BiblicalFonts_SBLHebrew.aspx
% https://www.sbl-site.org/educational/BiblicalFonts_SBLGreek.aspx
\usepackage{fth-lang}

% Für Literatur verwenden wir "fth-bib".
\usepackage[authorsc]{fth-bib}
% Hier muss der Pfad zur .bib Datei angegeben werden:
% TODO Micha: Bib-Datei ebenfalls ablegen oder aus Beispiel entfernen
\addbibresource{~/Nextcloud/Literatursammlung/lit.bib}

\author{Micha Piertzik}
\title{Abstract: Gottesfurcht im Alten Testament}
\subtitle{Eine Frame-semantische Untersuchung der Wurzel \heb{ירא}}

\begin{document}

\maketitle


Was bedeutet Gottesfurcht im Alten Testament? Zentral für diese Frage ist die hebräische Wurzel \heb{ירא}, deren Derivate am häufigsten verwendet werden, um dieses Konzept im Alten Testament auszudrücken. In der Forschungsgeschichte wurden verschiedene Deutungen von Gottesfurcht vorgebracht. Während die klassische Position des späten 20. Jahrhunderts davon ausgeht, es seien eher bestimmte Handlungen (kultische Verehrung oder sittliches Verhalten) gemeint, schlagen andere vor, es gehe um die Emotion Furcht mit Gott als Objekt derselben.

Um die Bedeutung von \heb{ירא} zu ergründen habe ich im Rahmen meiner B.A.-Arbeit eine Frame-semantische Analyse des Gebrauchs der Wurzel im biblischen Hebräisch durchgeführt, angelehnt an die Methode von Carsten Ziegert.
Um die Semantik zu bestimmen, wurde ein Frame für den \enquote{alltäglichen} Gebrauch der Wörter rekonstruiert.
Dieser ließ sich auf weniger alltägliche Texte, wie die \enquote{Furcht} vor Eltern oder Königen und Propheten übertragen und wurde schließlich auf die \enquote{Furcht} vor Gott in Dtn~6,13 (\heb{אֶת־יְהוָ֧ה אֱלֹהֶ֛יךָ תִּירָ֖א וְאֹתֹ֣ו תַעֲבֹ֑ד וּבִשְׁמֹ֖ו תִּשָּׁבֵֽעַ׃}) angewandt.
Die Arbeit wurde durch einen Ausblick auf das NT anhand von Mt~10,28 (\grk{Καὶ μὴ ⸀φοβεῖσθε ἀπὸ τῶν ἀποκτεννόντων τὸ σῶμα, τὴν δὲ ψυχὴν μὴ δυναμένων ἀποκτεῖναι· ⸁φοβεῖσθε δὲ μᾶλλον τὸν δυνάμενον καὶ ψυχὴν καὶ σῶμα ἀπολέσαι ἐν γεέννῃ.}) abgerundet.

Die Ergebnisse diese Arbeit vorzustellen, ist Ziel des Vortrags.
Es wird nötig sein, sich an verschiedenen Stellen zu beschränken.
Dabei sind meine zwei Hauptanliegen, die Methode verständlich zu machen und die theologische Diskussion um das Konzept Gottesfurcht anzuregen.

% Zum Schluss wollen wir noch ein paar Literaturverweise angeben. Dabei kann auch \heb{} oder \grk{} im Titel verwendet werden.
\nocite{Becker1965}
\nocite{Fuhs1982}
\nocite{Kipfer2016}

\printbibliography

\end{document}

