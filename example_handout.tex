% scrartcl ist eine geeignete Klasse für bspw. Handouts.
\documentclass{scrartcl}

% Wir wollen griechisch und hebräisch nutzen, daher "fth-lang".
% Voraussetzung dafür ist, dass die Fonts "SBL Hebrew" "SBL Greek" installiert sind!
% https://www.sbl-site.org/educational/BiblicalFonts_SBLHebrew.aspx
% https://www.sbl-site.org/educational/BiblicalFonts_SBLGreek.aspx
\usepackage{fth-lang}

% Für Literatur verwenden wir "fth-bib".
\usepackage[authorsc]{fth-bib}
% Hier muss der Pfad zur .bib Datei angegeben werden:
\addbibresource{literature.bib}

\author{Max Mustermann}
\title{Ein Handout}
\subtitle{Mit sollten Sprachen \heb{ירא} und \grk{λόγος}}

\begin{document}

\maketitle

Hier folgt der tolle Text des Handouts, inkl. Bibelstellen in Originalsprachen, z.~B. Dtn~6,13 (\heb{אֶת־יְהוָ֧ה אֱלֹהֶ֛יךָ תִּירָ֖א וְאֹתֹ֣ו תַעֲבֹ֑ד וּבִשְׁמֹ֖ו תִּשָּׁבֵֽעַ׃}).
Es braucht natürlich auch NT, s. Mt~10,28 (\grk{Καὶ μὴ ⸀φοβεῖσθε ἀπὸ τῶν ἀποκτεννόντων τὸ σῶμα, τὴν δὲ ψυχὴν μὴ δυναμένων ἀποκτεῖναι· ⸁φοβεῖσθε δὲ μᾶλλον τὸν δυνάμενον καὶ ψυχὴν καὶ σῶμα ἀπολέσαι ἐν γεέννῃ.}).



% Zum Schluss wollen wir noch ein paar Literaturverweise angeben. Dabei kann auch \heb{} oder \grk{} im Titel verwendet werden.
\nocite{aust}
\nocite{fuhs}

\printbibliography

\end{document}

