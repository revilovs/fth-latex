% Beim Laden der documentclass können zwar Schrift- und Papiergröße angegeben werden, allerdings überschreibt das Paket fth-lsa
\documentclass{scrreport}

\usepackage[ngerman]{babel}

\usepackage[thesis]{fth/fth-lsa}
% Verfügbare Option bound für gebundene Arbeiten (breiterer Rand)
% \usepackage[bound]{fth/fth-lsa}
% Verfügbare Option thesis für Abschlussarbeiten (andere Titelseite und Fußnoten kapitelweise nummeriert)
% Verfügbare Option title-style=sc verwendet für Titel und Untertitel auf dem Deckblatt Kapitälchen antatt Großbuchstaben

\usepackage{lipsum}

\begin{document}

% \title, \author und \date müssen immer gesetzt sein, alle anderen Felder sind optional
\title{Beispiel für die Titelseite von Abschlussarbeiten}
\subtitle{nach der \enquote{Leitlinien für schriftliche Arbeiten an der FTH}}
\thesistype{Bachelorarbeit}
\department{Praktische Theologie}
\referee{Prof. Dr. X}
\secondreferee{Dr. Y}
\wordcount{12345}
\author{Martin Muster}
\submissionplace{Gießen}
\date{25. November 2022}

\maketitle

\chapter{Ein Kapitel}
Dieses Kapitel dient nur dazu, zu überprüfen, ob die Fußnoten-Nummerierung \footnote{In Abschlussarbeiten soll nämlich kapitelweise nummeriert werden.} funktioniert. Deswegen kommt gleich das nächste Kapitel.

\chapter{Noch ein Kapitel}
Das hier ist noch ein Kapitel \footnote{Diese Fußnote sollte wieder die Nummer 1 bekommen}.

\lipsum \footnote{Das ist eine weitere Fußnote im zweiten Kapitel. Sie sollte die Nummer 2 erhalten. Fußnoten sollten immer mit Großbuchstabend beginnen und Punkt enden. Sind auch Fragen erlaubt?}



\end{document}
